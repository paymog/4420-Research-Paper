\documentclass[12pt]{report}
%\documentclass[12pt]{article}
% Change "article" to "report" to get rid of page number on title page

%************************************************************
% Package Imports
%************************************************************
\usepackage{fancyhdr}
\usepackage{abstract}
\usepackage{epsfig}
\usepackage{graphicx}       
\usepackage{appendix}       
\usepackage{makeidx}       
\usepackage{lastpage}          % references the page before
\usepackage{extramarks}       
\usepackage{tikz}              % this is to make fancy images/graphs/trees
\usepackage{xcolor}            % Text can be colored 
\usepackage{amsthm}            % Introduces theorems
\usepackage{amssymb}           % math symbols
\usepackage{amsmath}           % for align function 
\usepackage{mathrsfs}          % script math text
\usepackage[makeroom]{cancel}
\usepackage[ampersand]{easylist}


%************************************************************
% Linking parts of pdf i.e. with a "link"
%************************************************************
\usepackage[colorlinks=true,linkcolor=black,citecolor=cyan,urlcolor=blue,pagebackref=true]{hyperref}
%\usepackage{sidecap} for side captions

%****************************************************************************
%****************************************************************************
%	Global definitions
%****************************************************************************
%****************************************************************************

%**********************************************************
% In case you need to adjust margins:
%**********************************************************
\setlength{\textwidth}{6.0in}
\setlength{\textheight}{8.5in}
\setlength{\topmargin}{-0.4in}
\setlength{\evensidemargin}{.25in}
\setlength{\oddsidemargin}{.25in}
\setlength{\headsep}{0.25in}

%**********************************************************
% Don't know what this is for
%**********************************************************
\renewcommand{\baselinestretch}{1} 

\fancyfoot{}
\renewcommand{\headrulewidth}{0pt}
\rhead{\thepage}

\renewcommand{\absnamepos}{flushleft}
\renewcommand{\abstractnamefont}{\normalfont\bfseries\Large}

%****************************************************************************
%********************** CHANGE THESE ON EVERY DOCUMENT **********************
%****************************************************************************
%          These are things that are class and assignment specific
%****************************************************************************

\newcommand{\texTitle}{Project Propsal}
\newcommand{\dueDate}{Febuary 24, 2014}
\newcommand{\class}{COMP 4420}
\newcommand{\classTitle}{Advanced Design and Analysis of Algorithms}
\newcommand{\classInstructor}{Dr. S. Durocher}


%****************************************************************************
% Author "A" information
%****************************************************************************
\newcommand{\authorNameA}{Joshua\ Hernandez}
\newcommand{\studentNumberA}{6852561}
\newcommand{\emailA}{umhern23@myumanitoba.ca}

%****************************************************************************
% Author "B" information
%****************************************************************************
\newcommand{\authorNameB}{Paymahn\ Moghadasian}
\newcommand{\studentNumberB}{7XXXXXX}
\newcommand{\emailB}{umpaymah@myumanitoba.ca}

%****************************************************************************
% Gives pdf a signature
%****************************************************************************
% Fill in for important documents
%****************************************************************************
\hypersetup{
	pdfauthor={\authorNameA, \authorNameB},
	pdftitle={\texTitle, \class, \classTitle},
	pdfsubject={},
	pdfkeywords={}
}
\makeindex


%****************************************************************************
%****************************************************************************
% New definition of square root:
% it renames \sqrt as \oldsqrt
%****************************************************************************
\let\oldsqrt\sqrt
%****************************************************************************
% it defines the new \sqrt in terms of the old one
%****************************************************************************
\def\sqrt{\mathpalette\DHLhksqrt}
\def\DHLhksqrt#1#2{%
	\setbox0=\hbox{$#1\oldsqrt{#2\,}$}\dimen0=\ht0
	\advance\dimen0-0.2\ht0
	\setbox2=\hbox{\vrule height\ht0 depth -\dimen0}%
	{\box0\lower0.4pt\box2}
}
%****************************************************************************
%****************************************************************************

%****************************************************************************
%Some new commands to ease Dirac Notation
%****************************************************************************
\newcommand{\bra}[1]{\left\langle #1\right|}
\newcommand{\ket}[1]{\left|#1\right\rangle}
\newcommand{\braket}[2]{\left\langle #1\middle|#2\right\rangle}


%****************************************************************************
% This defines a running fraction.
%****************************************************************************
\newcommand*\rfrac[2]{{}^{#1}\!/_{#2}}


%****************************************************************************
% This defines a rounding functions
%****************************************************************************
\newcommand{\floor}[1]{\left\lfloor #1\right\rfloor}
\newcommand{\ciel}[1]{\left\lceil #1\right\rceil}

%****************************************************************************
% This defines total derivative operation
%****************************************************************************
\newcommand{\diff}[2]{ \frac{\text{d}#1}{\text{d}#2} }
\newcommand{\pdiff}[2]{ \frac{\partial #1}{\partial#2} }
%****************************************************************************
%****************************************************************************


%****************************************************************************
% Settings Based off Package Imports And User Defined Functions
%****************************************************************************
%\numberwithin{equation}{homeworkProblemCounter} % Equation Labeling
\everymath{\displaystyle}                       % Forces 'fixed' font in math equations
\allowdisplaybreaks[3]


%****************************************************************************
% Removes default section numbers
%****************************************************************************
\setcounter{secnumdepth}{0}         

%*********************************************
% Setup the header and footer
%*********************************************
\pagestyle{fancy}  
\lhead{\authorNameA,\ \authorNameB}
%\chead{\texTitle} 
\chead{} 
\rhead{\class}     
\lfoot{\lastxmark} 
\cfoot{}           
\rfoot{Page\ \thepage\ of\ \pageref{LastPage}}
\renewcommand\headrulewidth{0.4pt}            
\renewcommand\footrulewidth{0.4pt}  

%****************************************************************************
%****************************************************************************
% The end of the definitions for the document
% The start of the actual document
%****************************************************************************
%****************************************************************************
\begin{document}
	\pagenumbering{roman}
	%****************************************************************************
	% Title Stuff
	%****************************************************************************
	\begin{titlepage}
		\begin{center}
			~~ \\[0.4cm]

			{\LARGE \bfseries \texTitle }\\[0.4cm]

			~~ \\[1.5cm]

			% Author and supervisor
			\normalsize
			Authors: \\
			\authorNameA \ [\emailA] \\[.5cm]
			\authorNameB \ [\emailB] \\[.5cm]
			Instructor: \classInstructor \\[1.5cm]

			\class \\[.125cm]
			\classTitle \\[1.5cm]

			\vfill

			%{\today}
			{Due : \dueDate}
		\end{center}
	\end{titlepage}
	% Bottom of the page
	%****************************************************************************
	% Things Related to the beginning of the document
	%****************************************************************************
	%\tableofcontents
	\newpage
	\pagenumbering{arabic}
	%****************************************************************************
	% DUBUGIN TIPS
	%****************************************************************************
	% This is used to trace down (pin point) problems
	% in latexing a document:
	% \tracingall
	%****************************************************************************
	%****************************************************************************
	% Begin actual document content here
	%****************************************************************************

	%****************************************************************************
	% written preliminary project proposal details 
	%****************************************************************************
	%• 1–2 pages
	%• Briefly describe and motivate the topic you have selected.
	%• Describe the format of your proposed project.
	%• Provide a partial list references on the selected topic.
	%• Include a brief list of milestones and dates by which you plan to complete these.
	%****************************************************************************

	\section{Project Proposal}
	
	Sorting is a well studied topic in algorithms, in particular quicksort has been extensivly studied. 
	The quicksort is particularly interesting because it is a comparatively simple algorithm which has an average asymptotic runtime equal to that of the mergesort and heapsort without any additional memory requirements; quicksorts can be done in place without additional time or code complexity. 
	A major drawback of the quicksort is the it has a worst case complexity of $O(n\log n)$ unlike the mergesort or heapsort\cite{sedgewick1977analysis}.

	Aumuller and Dietzfelbinger published a paper in ICALP 2013 titled "Optimal Partitioning for Dual Pivot Quicksort" where they explored the asymptotic runtime of various quicksort along with their respective partitioning algorithms\cite{Aumuller:2013:OPD:2525857.2525862}. 
	Aumuller et al. mathematically demonstrated that dual pivot quicksorts are asymptotically lower bound by $1.8n\log n + o(n\log n)$ time where $n$ is the number of elements to be sorted. They generalized the math to describe any dual pivot quicksort algorithm and verified the asymptotic lower bounds of the classic quicksort and Yaroslavskiy's variation\cite{Wild:2012:ACA:2404160.2404231}. 

	We plan to first verify Aumuller's results and secondly to expand the scope of the original paper by mathematically describing all multi-partition variations. We will implement quicksort using various partitioning algorithms, again using varying number of partitions. We hope to determine if varying the number of paritions will yeild to a more efficent quicksort algorithm.

	The goal of this project is to investigate multi-paritioned quicksort, determine an optimal number of parititions and bring to light any of the challenges that arise from the mathematical analysis of multi-partition quicksorts. We're particularly interested in determining whether having more than two parititons provides any theoretical or experimental benefits.

	The following is an estimated list of milestones we wish to accomplish on a weekly basis :
	% Estimated time table
	\begin{itemize}
		\item Week 1 (Week of March 2 to 8)
		\begin{itemize}
			\item Implement basic quicksort
			\begin{itemize}
				\item Pivot Picking
				\begin{itemize}
					\item first element as pivot
					\item median of first, middle, and last
					\item Find others
				\end{itemize}
				
				\item Partitioning algorithm
				\begin{itemize}
					\item The basic one
				\end{itemize}
				%\item Each version has an insertion sort flag on arrays with small size
				\item Test functionaliy on small arrays
			\end{itemize}
			
			\item Implement 2 parition quicksort
			\begin{itemize}
				\item Piviot picking
				\begin{itemize}
					\item first and last elements
					\item two middle elements of entires from 5 entries
					\item Test with 4 entries as well
				\end{itemize}
				\item Paritioning Algorithm
				\begin{itemize}
					\item Basic Paritioning
					\item make smalls then bigs
					\item flag paritioning algorithm
				\end{itemize}
				%\item Each version has an insertion sort flag on arrays with small size
				\item Test functionality on small arrays
			\end{itemize}
		\end{itemize}
			
		\item Week 2 (Week of March 9 to 15)
		\begin{itemize}
			\item Implement 3 and 4 parition quicksorts (more if time allows)
			\begin{itemize}
				\item Piviot picking
				\begin{itemize}
					\item Basic Pivot selecion
					\item Look and implement other piviot selection algorithms
				\end{itemize}
				\item Parition algorithm
				\begin{itemize}
					\item Basic Parition 
					\item Look for parition algorithms
				\end{itemize}
				%\item Each version has an insertion sort flag on arrays with small size
				\item Test functionallity on small arrays
			\end{itemize}
		\end{itemize}
		
		\item Week 3 (Week of March 16 to 22)
		\begin{itemize}
			\item Run several experinments using all the quicksort algoirthms implemented
			\item Run several arrays sizes
			\item Array 'types'
			\begin{itemize}
				\item sorted arrays
				\item reverse sorted arrays
				\item random arrays
				\item partially random arrays
			\end{itemize}
			\item Preliminary analysis of data
		\end{itemize}
		\item Week 4 (Week of March 23 to 29)
		\begin{itemize}
			\item Analyze data
			\item Make Presentation    
		\end{itemize}
		\item Week 5 (Week of March 30 to April 5)
		\begin{itemize}
			\item Continue Analyze data
			\item Write paper
		\end{itemize}
	\end{itemize}

	%****************************************************************************
	% End of the contents of the document
	% Items related to end of document
	%****************************************************************************
	\newpage
	%\addcontentsline{toc}{section}{References}
	\bibliographystyle{ieeetr}
	\bibliography{refer}
	\newpage
	%\printindex
	%****************************************************************************
\end{document}